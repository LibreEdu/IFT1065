\section*{Question 4}
$f(n)$ est la somme des $n^3$ premiers termes. Il y a en tout $n^3$ termes.

\[f(n) = \underbrace{1 + 2 + 3 + \dots + n^3}_{n^3\ \text{termes}}\]

\[1 + 2 + 3 + \dots + n^3 \leq n^3 + n^3 + n^3 + \dots + n^3 = n^3 \times n^3 = n^6\]

On a donc $f(n) = \mathrm{O} (n^6)$.

Considérons la deuxième moitié de la somme, les termes allant de $\left\lceil \frac{n^3+2}{2} \right\rceil$ à $n^3$. On a :

\[1 + 2 + 3 + \dots + n^3 \geq \left\lceil\frac{n^3+2}{2}\right\rceil + \dots + n^3 -1 + n^3\]

Chacun des $\left\lceil \frac{n^3}{2} \right\rceil$ termes de droite est supérieur à $\left\lceil\frac{n^3+1}{2}\right\rceil$. On a donc :

\[ f(n) \geq \left\lceil\frac{n^3}{2}\right\rceil \cdot \left\lceil\frac{n^3+1}{2}\right\rceil \geq \frac{n^3}{2} \cdot \frac{n^3}{2} = \frac{n^6}{4}\]

$f(n) = \Omega (n^6)$

$f(n) = \mathrm{O} (n^6)$ et $f(n) = \Omega (n^6)$, donc  $f(n) = \Theta (n^6)$.