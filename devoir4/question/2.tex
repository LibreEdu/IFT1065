\newpage
\section*{Question 2}
On a 31 bâtiments, et si l’on veut les numéroter en ordre croissant sans avoir deux numéros consécutifs, alors on a :

\begin{center}
	\begin{tabular}{l@{ $\rightarrow$ }l}
		 Bâtiment 1  & numéro 1                                          \\
		 Bâtiment 2  & numéro 3                                          \\
		 Bâtiment 3  & numéro 5                                          \\
		\multicolumn{2}{c}{$\hfill\vdots\hfill\hfill\hfill\vdots\hfill\hfill$} \\
		Bâtiment $i$ & numéro $2 i - 1$                                  \\
		\multicolumn{2}{c}{$\hfill\vdots\hfill\hfill\hfill\vdots\hfill\hfill$} \\
		Bâtiment 30  & numéro 59                                         \\
		Bâtiment 31  & numéro 61
	\end{tabular}
	\end{center}

Mais, nous avons seulement 60~numéros. On a donc $n > k$ où $n$ est le nombre de maison et $k$ est le nombre de numéros possible. Alors par principe du pigeonnier on a forcément deux numéros consécutifs.