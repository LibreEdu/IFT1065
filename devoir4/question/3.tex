\section*{Question 3}
On va l’algorithme faire par récurrence. Pour cela, nous avons besoin d’une procédure qui permet d’extraire une chaine de caractère.

Convention :

\begin{tabular}{l@{ : }l}
	Opérateur d’égalité        & $=$          \\
	Opérateur d’affectation    & $\leftarrow$ \\
	Opérateur de concaténation & $+$
\end{tabular}

\subsection*{Algorithme SousChaine}
%https://ctan.math.ca/tex-archive/macros/latex/contrib/algorithm2e/tex/algorithm2e.sty
\SetKw{KwTo}{à}

\begin{procedure}
	\SetKwProg{procSousChaine}{Procédure SousChaine}{}{Fin\ SousChaine}
	
	\Res{Extrait une partie d’une chaine de caractères}
	\Entree{Une chaine de caractères, l’index du premier caractère, l’index du dernier élément}
	\Sortie{La chaine de caractères entre les deux index}
	\procSousChaine{(chaine, debut, fin)}{
		$sousChaine \leftarrow$ "" \;
		\Pour{$index\leftarrow$ debut \KwTo fin}{
			$sousChaine \leftarrow sousChaine + chaine[index]$ \;
		}
		\Retour $sousChaine$ \;
	}
\end{procedure}

\subsection*{Algorithme InverserChaine}
\begin{procedure}
	\SetKwProg{procInverserChaine}{Procédure InverserChaine}{}{Fin\ InverserChaine}
	
	\Res{Inverse les caractères d’une chaine de caractères}
	\Entree{Une chaine de caractères}
	\Sortie{La même chaine, mais inversée}
	\procInverserChaine{(chaine)}{
		$longueur \leftarrow len(chaine)$ \;
		\Si{longueur = 1}{
			\Retour $chaine$ \;
		}\SinonSi{longueur = 2}{
			\Retour $chaine[2] + chaine[1]$ \;
		}\Sinon{
			$dernierCaractere \leftarrow chaine[longueur]$ \;
			$premierCaractere \leftarrow chaine[1]$ \;
			$leReste \leftarrow SousChaine(chaine, 2, longueur-1)$ \;
			\Retour $ dernierCaractere + InversChaine(leReste) + premierCaractere$ \;
		}
		
	}
\end{procedure}