\section*{Question 1}
\emph{Montrez que \( A \subseteq B \Rightarrow A \cap C \subseteq B \cap C\). Pour cela réécrivez l’expression en utilisant quantificateurs et prédicats.}

\bigskip

On suppose que la proposition \( A \subseteq B \) est vraie. Pour tout x appartenant à l’intersection des ensembles A et C, est-ce que x appartient à l’intersection des ensembles B et C ?

\begin{equation}
	\forall x, \; x \in A \cap C \quad \equiv \quad \forall x, \; x \in A \; \wedge \; x \in C
	\label{q1-eq1}
\end{equation}

Or, nous avons supposé que \( A \subseteq B \), donc si x est un élément de l’ensemble A, alors x est un élément de l’ensemble B. La partie de droite de l’équation (\ref{q1-eq1}) peut donc se réécrire :

\[\forall x, \; x \in B \; \wedge \; x \in C\]

On a donc, \(\forall x, \; x \in B \cap C\). \quad $\square$

%\bigskip
%Cela dit, l’exercice n’était pas de démontrer cette implication, mais juste de \emph{réécrire} \og l’expression en utilisant quantificateurs et prédicats \fg{}.
%
%\begin{align*}
%	A \subseteq B \quad               & \equiv \forall x, \; x \in A \; \Rightarrow \; x \in B                                                                                                  &  \\
%	A \cap C \subseteq B \cap C \quad & \equiv \forall x, \; x \in A \; \wedge \; x \in C \; \Rightarrow \; x \in B \; \wedge \; x \in C \\
%	A \subseteq B \Rightarrow A \cap C \subseteq B \cap C & \equiv (\forall  x, \; x \in A \; \Rightarrow \; x \in B) \Rightarrow (\forall \vee x, \; x \in A \; \wedge \; x \in C \; \Rightarrow \; x \in B \; \wedge \; x \in C)
%\end{align*}
