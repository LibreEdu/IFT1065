\section*{Question 1}
\emph{Montrez que \( A \subseteq B \Rightarrow A \cap C \subseteq B \cap C\). Pour cela réécrivez l’expression en utilisant quantificateurs et prédicats.}

\bigskip

On suppose que la proposition \( A \subseteq B \) est vraie. Pour tout x appartenant à l’intersection des ensembles A et B, est-ce que x appartient à l’intersection des ensembles B et C ?

\begin{equation}
	x \in A \cap C \quad \equiv \quad x \in A \; \wedge \; x \in C
	\label{q1-eq1}
\end{equation}

Or, nous avons supposé que \( A \subseteq B \), donc si x est un élément de l’ensemble A, alors x est un élément de l’ensemble B. La partie de droite de l’équation (\ref{q1-eq1}) peut donc se réécrire :

\[x \in B \; \wedge \; x \in C\]

Par conséquent, x est un élément de \( B \cap C \). \quad $\square$