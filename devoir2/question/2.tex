\section*{Question 2}
\emph{Montrez que l’ensemble des nombres premiers est infini à l’aide du résultat décrit pour les nombres composés.}

\bigskip
Supposons que l’ensemble des nombres est fini et que sa cardinalité est $k$. Soit $q$ le nombre représenté par l’ensemble \( \{(p_1, 1), (p_2, 1), \dots (p_k, 1)\}\), \( q = p_1 . p_2 . p_3 \dots p_k\), la multiplication de tous les nombres premiers. Considérons le nombre \(r = q + 1\), \( r = p_1 . p_2 . p_3 \dots p_k + 1\).

\medskip

$r$ n’est pas divisible par $p_1$, car le reste de la division de $r$ par $p_1$ est 1. De même $r$ n’est pas divisible par $p_2$, $p_3$ \dots $p_k$. Soit r est un \og nouveau \fg{} nombre premier, soit c’est la multiplication de nombres premiers qui ne sont pas dans la liste $p_1, p_2 \dots p_k$. Il s’agit donc d’un ensemble infini.

\begin{flushright}
	$\Box$
\end{flushright} 