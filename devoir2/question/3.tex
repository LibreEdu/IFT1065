\section*{Question 3}
\noindent
\emph{Cas de base : 1 carré }
\begin{align*}
	nombre\ de\ coupe & = n -1  \\
	                  & = 1 - 1 \\
	                  & = 0
\end{align*}
effectivement si nous avons un carré, nous devons le diviser zéro fois.

\bigskip
\emph{Étape inductive : }

Posons un rectangle de k morceaux tel que $ 1 \leq k < n$. Le rectangle k est de base k\textsubscript{2} et de hauteur k\textsubscript{1}, donc

\begin{align*}
k = k_1 \times k_2
\end{align*}

Si nous effectuons une coupure au rectangle k d'une valeur j. Nous avons deux nouveaux rectangles :
\begin{align*}
&premier\ rectangle = k_1 \times j\\
&deuxième\ rectangle = k_1 \times (k_2 - j)
\end{align*}

Ainsi, pour calculer le nombre de coupure à faire pour ces nouveaux rectangles : 
\begin{align*}
&(k_1 \times j) - 1 + (k_1(k_2 \times j)-1) + 1\\
&= k_1 j - 1 + k_1 k_2 - k_1 j - 1 + 1\\
&= k_1 k_2 - 1\\
&= k - 1
\end{align*}
\begin{flushright}
	$\Box$
\end{flushright}