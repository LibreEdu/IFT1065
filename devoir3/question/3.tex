\section*{Question 3}
\noindent
\emph{Transitivité :}

Si$\ R_1 = \{(1,1), (2,2), (1,2), (2,1)\}\ et\ R_2 = \{(1,1), (5,5), (1,5), (5,1)\}$. Donc, $\ R_1 \cup R_2 = \{(1,1), (2,2), (1,2), (2,1), (5,5), (1,5), (5,1)\}$. L'union des  deux relations n'est pas transitive. Puisque$\ (2,1) \in R_1\ et\ (1,5) \in R_2,\ mais\ (2,5) \notin R_1 \cup R_2$.

\bigskip

\emph{Réflexivité :}

$\forall x\in X$,$\ xR_1x \in R_1$ parce que $R_1$ est réflexif. De même $xR_2x \in R_2$ parce que $R_2$ est réflexif. Finalement, on a que$\ xR_1x \in R_1$ et$\ xR_2x \in R_2$. Donc,$\ (x,x) \in R_1\cup R_2$.

\bigskip

\emph{Symétrie : }

$\forall x,y \in X$, si$\ (x,y)\ R_1$ alors$\ (y,x) \in R_1$ parce que $R_1$ est symétrie. De même, si$\ (x,y) \in R_2$ alors$\ (y,x) \in R_2$ parce que $R_2$ est symétrie. Si$\ (x, y) \in R_1$ ou $R_2$, alors$\ (y,x) \in R_1$ ou$\ R_2$. Donc,$\ (x, y)$ et
$\ (y,x) \in R_1 \cup R_2$.