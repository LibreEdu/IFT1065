\section*{Question 4}
%http://joshua.smcvt.edu/latex2e/_005cparindent-_0026-_005cparskip.html
\setlength\parindent{17pt}
\subsection*{(a)}
On va raisonner du point de vue de la place, c’est la place qui choisit la personne. La première place elle choisit une personne parmi huit, la deuxième place choisit une personne parmi sept et ainsi de suite. Le nombre de choix possible est $ 8 \times 7 \times 6 \times 5 = 1680$.

Raisonnement plus mathématique : l’ordre est important, c’est le nombre de 4-permutations d’un ensemble de 8 personnes.

\[P(8,4) = \frac{8!}{(8-4)!}\]