\section*{Question 4}
%http://joshua.smcvt.edu/latex2e/_005cparindent-_0026-_005cparskip.html
\setlength\parindent{17pt}
\subsection*{(a)}
On va raisonner du point de vue de la place, c’est la place qui choisit la personne. La première place elle choisit une personne parmi huit, la deuxième place choisit une personne parmi sept et ainsi de suite. Le nombre de choix possible est $ 8 \times 7 \times 6 \times 5 = 1680$.

Raisonnement plus mathématique : l’ordre est important, c’est le nombre de 4-permutations d’un ensemble de 8 personnes.

\[P(8,4) = \frac{8!}{(8-4)!}\]

\subsection*{(b)}
On prend Bob, on a le choix entre 4 places. Pour la place suivante on a le choix entre 7 personnes, puis 6 pour la place suivante et 5 pour la dernière place. On a donc $ 4 \times 7 \times 6 \times 5 = 840$ arrangements qui contiennent Bob.

\subsection*{(c)}
On prend Bob, on a le choix entre 4 places. Puis on prend Céline, on a le choix entre 3 place. Pour la place suivante on a le choix entre 6 personnes, puis 5 pour la dernière place. On a donc $ 4 \times 3 \times 6 \times 5 = 360$ arrangements qui contiennent Bob.

\subsection*{(d)}
On peut avoir ou Bob, ou Céline ou les deux. C’est tous les arrangements sauf celles qui contiennent ni Bob ni Céline.

Arrangements contenant ni Bob ni Céline : $ 6 \times 5 \times 4 \times 3 = 360$. C’est le même raisonnement que à la question (a).

Arrangements contenant Bob ou Céline $=$ tous les arrangements de la question (a) $-$  arrangements contenant ni Bob ni Céline. On a donc $1680-360 = 1320 $ arrangements qui contiennent Bob ou Céline.