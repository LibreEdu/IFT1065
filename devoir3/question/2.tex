\section*{Question 2}
\emph{Soit R, une relation symétrique définie sur X. Soit $R^1 = R\ et\ R^n = R^{n-1} \circ R = R \circ R^{n-1}\ pour\ n \geq 2.\ Montrer\ que\ pour\ tout\ n\geq1,\ R^n\ est\ une\ relation\ symétrique$.}

\bigskip

\emph{cas de base: }
$n = 1$\\
$R^1$ est symétrique par définition.

\bigskip

\emph{Étape inductive : }\\
Par hypothèse $R^n$ est symétrique, est-ce que $R^{n+1}$ est symétrique?
\bigskip

$xR^{n+1}y = xR^n \circ Ry$\\
On a $xR^nt$ et $tRy$. Parce que $R^n$ et $R$ sont symétriques on a $tR^nx$ et $yRt$. Donc, on a $xR^n \circ Ry \equiv xR^{n+1}y$ et $yR \circ R^nx \equiv yR^{n+1}x$, alors $R^{n+1}$ est symétrique.